\documentclass{homework}


\title{Módulo 2 - Lista de Exercícios 2 (2021/1 REMOTO)}
\author{Carlos Bravo\\ 119136241}
\date{Setembro 2021}

\begin{document}

\maketitle
%Questão 1
\exercise*
(a) A parte realizada pela main já está comentada, então passando para o algoritmo concorrente.

A barreira é uma forma de bloquear todas as threads até que estejam no mesmo ponto do código, ela possui um contador que vai somando e bloqueando a thread. Quando o número de threads bloqueadas é nthreads-1 a barreira é liberada e todas as threads podem seguir, zerando o contador.

A ideia do algoritmo implementado na tarefa é fazer uma "soma logarítmica", primeiro somando o elemento anterior, depois os 3 anteriores, depois os 7 anteriores e assim até somar todos em cada posição. A variável salto irá percorrer o vetor em potências de 2. Se o id está depois do corte realizado, irá pegar o valor da posição id-salto, esperar as outras threads, modificar o valor atual somando e esperar as outras threads novamente, passando pro próximo corte.

(b) Sim, esse algoritmo funciona pois o valor na posição $i$ depende do valor da posição $i-1$, mas ambos possuem a soma até a posição $i-2$. Análogo a como em um fatorial é possível reaproveitar valores anteriores. Dessa forma o algoritmo primeiro irá somar a posição anterior. Para obter a soma das 3 anteriores, a posição $i-1$ já foi somada no primeiro passo, e o valor que se encontra na posição $i-2$ é a soma dos valores $i-2$ e $i-3$. Assim, é possível pegar esse valor e obter a soma das 3 posições anteriores. Seguindo esse mesmo raciocínio, é possível pegar as somas nas potências de 2 anteriores a uma posição.

(c) A primeira barreira tem o objetivo de garantir que cada thread consiga pegar um valor anterior antes dele ser alterado no próximo passo. A segunda barreira tem o objetivo de garantir que cada thread consiga alterar o valor antes dele ser lido por outra thread no próximo passo. Se remover qualquer uma das duas barreiras poderá haver condições de corrida, com threads lendo valores durante uma alteração.

%Questão 2
\exercise*
\begin{lstlisting}[language=C]
// Condicao de ser multiplo de 100
pthread_cond_t cond; 
pthread_mutex_t mutex;
long long int contador = 0;

void *T1 (void * arg){
  while(1){
    FazAlgo(contador);
    // Trava para poder alterar a variavel global
    pthread_mutex_lock(&mutex);  
    contador++;
    // Se for multiplo libera a condicao e espera
    if(contador % 100 == 0){ 
      pthread_cond_signal(&cond);
      pthread_cond_wait(&cond, &mutex);
    }
    pthread_mutex_unlock(&mutex);
  }
}

void *T2(void *arg){
  while (1){
    // Trava para poder ler a variavel global sem ser alterada
    pthread_mutex_lock(&mutex); 
    // Se for multiplo imprime e libera a condicao para T1
    if(contador % 100 == 0){ 
      printf("%lld\n",contador);
      pthread_cond_signal(&cond);
    }
    // Espera a condicao estar liberada novamente
    pthread_cond_wait(&cond, &mutex); 
    pthread_mutex_unlock(&mutex);
  }
}
\end{lstlisting}
É necessário um if em T2 ao invés de começar com wait, mesmo com o if em T1, pois se a variável contador chegar a 100 e T2 não tiver começado, irá direto para o wait enquanto T1 já está esperando, entrando num deadlock.

%Questão 3
\exercise*
(a) As tarefas a serem executadas são salvas em uma lista encadeada. As threads são criadas e ficam em loop, esperando uma tarefa para realizar. Se a thread está livre e há uma tarefa na lista, começa a executar a primeira. Quando todas as tarefas são realizadas e o pool recebe o aviso de shutdown, as threads são finalizadas.

(b) O erro ocorre quando as tarefas são finalizadas mas não há o aviso de shutdown antes de todas as threads serem pausadas. Pode ser que após a criação das tarefas, a thread da main seja pausada e as threads do pool consigam realizar todas as tarefas, entrando em modo de espera. No entanto, quando a main retornar e der o aviso de shutdown, irá esperar as threads acabarem pelo $join$, no entanto, elas estão travadas, então o código não irá terminar. Para solucionar esse problema, quando der o aviso de shutdown é possível notificar todas as threads para voltarem a funcionar, saindo da condição de espera e indo para a condição de finalização.
\begin{lstlisting}[language=Java]
public void shutdown() {
    synchronized(queue) {
      this.shutdown=true;
      queue.notifyAll();
    }
    for (int i=0; i<nThreads; i++)
      try { threads[i].join(); }
      catch (InterruptedException e) {return;}
}
\end{lstlisting}

%Questão 4
\exercise*
(a) A inanição pode ocorrer quando há sempre leitores lendo. Um caso em que isso ocorre é quando o tempo de leitura é muito longo ou há muitas threads leitoras. Assim, antes de um leitor terminar de ler, outros leitores começarão, nunca permitindo uma brecha para os escritores.

(b)
\begin{lstlisting}[language=Java]
public class LE{
  private int leit; // Quantidade de leitores lendo
  private int escr; // Quantidade de escritores escrevendo
  private int escrQuer; // Quantos escritores querem escrever

  public LE(){
    this.leit = 0;
    this.escr = 0;
    this.escrQuer = 0;
  }

  // Entrada de leitores
  public synchronized void EntraLeitor (int id) {
    try {
      // Enquanto tiver um escritor querendo escrever, se bloqueia
      while (this.escrQuer > 0) {
        System.out.println ("le.leitorBloqueado("+id+")");
        wait();
      }
      this.leit++; // Leitor lendo
      System.out.println ("le.leitorLendo("+id+")");
    } catch (InterruptedException e) { }
  }

  // Saida de leitores
  public synchronized void SaiLeitor (int id) {
    this.leit--; // Saiu o leitor
    // Se nao tiver mais leitores, libera apenas um escritor
    if (this.leit == 0)
      this.notify();
    System.out.println ("le.leitorSaindo("+id+")");
  }

  // Entrada de escritores
  public synchronized void EntraEscritor (int id) {
    try {
      this.escrQuer++; // Escritor quer escrever
      // Enquanto tiver alguem lendo ou escrevendo, se bloqueia
      while ((this.leit > 0) || (this.escr > 0)) {
        System.out.println ("le.escritorBloqueado("+id+")");
        wait(); 
      }
      this.escr++; // Escritor escrevendo
      System.out.println ("le.escritorEscrevendo("+id+")");
    } catch (InterruptedException e) { }
  }

  // Saida de escritores
  public synchronized void SaiEscritor (int id) {
    this.escr--; // Escritor terminou
    this.escrQuer--; // Escritor nao quer mais escrever
    notifyAll();
    System.out.println ("le.escritorSaindo("+id+")");
  }
}

\end{lstlisting}
\end{document}
